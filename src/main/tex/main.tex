%! Author = Andrew Selvia
%! Date = 2022.4.26

\documentclass[conference]{./IEEEtran/IEEEtran} % https://www.ctan.org/pkg/ieeetran
\usepackage{cite}
\usepackage[hidelinks]{hyperref}
\usepackage{csquotes} % https://tex.stackexchange.com/a/36813
\usepackage{mathtools} % https://tex.stackexchange.com/a/96353
\usepackage{amsmath}
\usepackage{amsfonts}

\begin{document}

    \title{DrQ-v2 Ablation Study}

    \author{
        \IEEEauthorblockN{Paul Mello}
        \IEEEauthorblockA{\textit{Department of Artificial Intelligence} \\
        \textit{San José State University}\\
        San José, California \\
        paul.mello@sjsu.edu}
        \and
        \IEEEauthorblockN{Andrew Selvia}
        \IEEEauthorblockA{\textit{Department of Software Engineering} \\
        \textit{San José State University}\\
        San José, California \\
        andrew.selvia@sjsu.edu}
        \and
        \IEEEauthorblockN{Hyelim Yang}
        \IEEEauthorblockA{\textit{Department of Artificial Intelligence} \\
        \textit{San José State University}\\
        San José, California \\
        hyelim.yang@sjsu.edu}
    }

    \maketitle

    \begin{abstract}
        TODO
    \end{abstract}

    \begin{IEEEkeywords}
        reinforcement learning, actor-critic
    \end{IEEEkeywords}

    \section{Introduction}\label{sec:introduction}
    Example citation: DrQ-v2 paper~\cite{yarats2021drqv2}.

    Example citation: original paper~\cite{yarats2021image}.

    Example quotation: \enquote{This is important.}

    Example inline math with vector notation: \(\mathbf{w} = [1, 2, 3]\).

    Example math block:

    \[w = w - \eta \nabla f(w).\]

    \section{Problem Definition}\label{sec:problem-definition}

    \section{Methodology}\label{sec:methodology}

    TODO

    \section{Experiments}\label{sec:experiments}

    TODO

    \section{Conclusion}\label{sec:conclusion}

    TODO

    \bibliography{main}
    \bibliographystyle{ieeetr}
\end{document}
