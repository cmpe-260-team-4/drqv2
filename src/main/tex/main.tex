%! suppress = EscapeUnderscore
%! Author = Andrew Selvia
%! Date = 2022.4.26

\documentclass[conference]{./IEEEtran/IEEEtran} % https://www.ctan.org/pkg/ieeetran
\usepackage{cite}
\usepackage[hidelinks]{hyperref}
\usepackage{csquotes} % https://tex.stackexchange.com/a/36813
\usepackage{mathtools} % https://tex.stackexchange.com/a/96353
\usepackage{amsmath}
\usepackage{amsfonts}
\usepackage{tikz} % To generate the plot from csv

\usepackage{pgfplots}
\pgfplotsset{width = 10cm, compat = 1.9}

%% We will externalize the figures
\usepgfplotslibrary{external}
\tikzexternalize[prefix=figures/]

\begin{document}

    \title{DrQ-v2 Ablation Study}

    \author{
        \IEEEauthorblockN{Paul Mello}
        \IEEEauthorblockA{\textit{Department of Artificial Intelligence} \\
        \textit{San José State University}\\
        San José, California \\
        paul.mello@sjsu.edu}
        \and
        \IEEEauthorblockN{Andrew Selvia}
        \IEEEauthorblockA{\textit{Department of Software Engineering} \\
        \textit{San José State University}\\
        San José, California \\
        andrew.selvia@sjsu.edu}
        \and
        \IEEEauthorblockN{Hyelim Yang}
        \IEEEauthorblockA{\textit{Department of Artificial Intelligence} \\
        \textit{San José State University}\\
        San José, California \\
        hyelim.yang@sjsu.edu}
    }

    \maketitle

    \begin{abstract}
        
    Modern approaches to reinforcement learning (RL) are notorious for their extensive training times and complexity due, in
    part, to the necessary robustness of these systems in an uncertain environment space.\ As the problem space complexity
    continues to rise, more efficient systems must be developed to deal with the exponential rise in training times.\ DrQ-v2\cite{DrQv2}
    is the second iteration of an off-policy actor-critic sampling algorithm which aims to efficiently model complex continuous
    systems with the help of data augmentation and notably uses single gpu architecture.\ DrQ-v2 focuses on solving visually
    continuous control through direct pixel observation.\ This state-of-the-art algorithm is the first model free approach to be
    capable of direct pixel observation and solve humanoid movement tasks in record time.\ In this paper we will highlight
    important components which contribute to its efficiency and ingenuity.\ We will then demonstrate our experiments which help
    measure the effectiveness of this models's hyperparameters.\ Finally we will present the results of our ablation study which
    demonstrate that DrQ-v2 is an exceptionally well tuned model and one that may democratize the reinforcement learning field.\

    \end{abstract}

    \begin{IEEEkeywords}
        reinforcement learning, actor-critic, PPO
    \end{IEEEkeywords}

    \section{Introduction}\label{sec:introduction}

    %Example citation: DrQ-v2 paper~\cite{yarats2021drqv2}.
    %Example citation: original paper~\cite{yarats2021image}.
    %Example quotation: \enquote{This is important.}
    %Example inline math with vector notation: \(\mathbf{w} = [1, 2, 3]\).
    %Example math block:
    %\[w = w - \eta \nabla f(w).\]

    Efficient sampling of high-dimensional data has been a long standing problem in the field of RL.\ Through recent advances in
    sampling and data augmentation, DrQ-v2 has been able to solve complex humanoid tasks in a fraction of the time other models
    would need.\ DreamerV2 \cite{DreamerV2} is one of these state-of-the-art models which has incorporated these recent advances.\
    Despite this, it trains 4x slower than DrQ-v2 and requires a multi-gpu architecture.\ To understand this discrepancy in training
    times we must look at the design decisions and the components which make up the model.\ It uses a range of different methods
    to achieve its efficiency including a decay scheduler, data augmentation, sampling, improved replay buffer, and a deep
    deterministic policy gradient (DDPG).\
    The decay scheduler acts as a way of controlling the exploration rate of the model.\ Data augmentation in DrQ-v2 uses a litany
    of methods including image flipping and cropping to ensure the model is properly generalizing the problem space.\ Sampling is
    used to reduce the computational load by extracting 3 consecutive pixel observations at a time.\ This sampling efficiently
    models the policy by capturing the minimal amount of information necessary to make improvements.\ The replay buffer has been
    significantly updated  by expanding its size.\ This increase has directly improved performance by 3.5x due to an increase in
    explored policy paths. Finally, and most notably, DrQ-v2 uses a DDPG to manage the rate of reward propagation.\ This process
    for continuous systems  updates the critic by comparing the error from its network to that of the actor network error.\
    While this model is state-of-the-art, with a range of optimizations, there appears to be further room for improvement in a
    range of minor changes.\ The aim of this paper is thus to explore the inner mechanisms of DrQ-v2 and present additional
    optimizations that may incur performance benefits to our training time and convergence.\ We will also explore methods such
    as standard deviation tuning and learning rate scheduling to meet these goals.\


    \section{Problem Definition}\label{sec:problem-definition}

    The stunning success of DrQ-v2, when compared to other state-of-the-art methods, warrants further exploration into its
    components.\ During this initial exploration we found certain design decisions, such as hyperparameters and optimizers, which
    appeared to have been left unoptimized.\ With this understanding we began to formulate experiments we could conduct on DrQ-v2
    which may offer some improvement to the base model.\ The following sections will demonstrate our attempts at improving
    training times and convergence rates given the already highly optimized design.\ We use San Jose State University’s HPC,
    which uses an Intel Xeon E5-2660 and an NVIDIA Tesla P100, to conduct our tests over 48 hour time intervals.\ Due to the
    computational efficiency of DrQ-v2, we expect this hardware to induce convergence.\

    \section{Methodology}\label{sec:methodology}

    DrQ-v2 has been demonstrated to achieve state-of-the-art results on a number of mujoco environments.\ As a result it is
    important to dissect the model apart and understand how it operates.\ In order to do this we must first explore the structure
    of the code.\ Once the code structure is understood, we will begin the ablation study by testing DrQ-v2 in multiple
    environments.\ We test a number of mujoco environments which are challenging enough to test the abilities of our model while
    maintaining the prospect of convergence within our 48 hour window.\ After selecting a promising environment, we will set a
    control run and tweak components in an iterative process.\ This will cleanly communicate if any adjustments to the model's
    components will provide additional performance gains not present in the initial implementation of DrQ-v2.\

    \section{Experiments}\label{sec:experiments}

    When studying the code and reading through the paper we noticed a few interesting points where optimization could be done.
    Particularly we focused on the learning rate, the optimizer, and the standard deviation scheduler.\ These components were
    selected in part due to their limited impact on every other component.\ To begin our experiments we must first find a mujoco
    environment which balanced complexity and convergence.\ Following this we began by creating a control test and attempting to
    improve from that starting point.\ We tweak many different hyperparameters, in a multitude of ways, in an attempt to build a
    stronger model.\ In the following sections we detail the theory and the results of these experiments.\

    Mujoco contains many different environments with varying degrees of difficulty which we can use to assess the abilities of DrQ-v2.\
    We began by testing random environments before noticing that we would not be able to see convergence in environments with
    considerable dimensionality due to resource restrictions.\ This is because the results published in the DrQ-v2 paper uses a
    better GPU and likely has updated software.\ With our implementation using SJSU’s HPC we are limited by what is available on
    these systems.\ Figure 1 demonstrates the rewards of these initial experiments, while Figure 2 demonstrates the actor loss at
    each step.\

    % Figure 1
    \begin{tikzpicture}
        \begin{axis}[
        scale = .8,
        title = {Mujoco Environments: Reward},
        xlabel = {Episode},
        ylabel = {Reward},
        xmin = 0, xmax = 3200,
        ymin = 0, ymax = 1050,
        xtick = {0,800,1600,2400,3200},
        ytick = {0,150,300,450,600,750,900,1050},
        legend pos = south east,
        ymajorgrids = true,
        grid = major,
        mark size = .7pt,
        axis line style = ultra thin,
        legend style = {nodes = {scale = 0.75, transform shape}},
        legend entries = {$Walker Run$, $Cheetah$, $Walker Walk$, $Humanoid Walk$, $Quadruped$,
                        $Walker Stand$, $Cartpole Balance$, $Pendulum Swingup$}
        ]

    \addplot table [x = episode, y = episode_reward, col sep = comma]{experiments/203408-walker-run-eval-control.csv};
    \addplot table [x = episode, y = episode_reward, col sep = comma]{others/031508-cheetah-run-eval.csv};
    \addplot table [x = episode, y = episode_reward, col sep = comma]{others/154418-walker-walk-eval.csv};
    \addplot table [x = episode, y = episode_reward, col sep = comma]{others/175732-humanoid-walk-eval.csv};
    \addplot table [x = episode, y = episode_reward, col sep = comma]{others/180549-quadruped-eval.csv};
    \addplot table [x = episode, y = episode_reward, col sep = comma]{others/190918-walker-stand-eval.csv};
    \addplot table [x = episode, y = episode_reward, col sep = comma]{others/191540-cartpole-balance-eval.csv};
    \addplot table [x = episode, y = episode_reward, col sep = comma]{others/201234-pendulum-swingup-eval.csv};

    \end{axis}
    \end{tikzpicture}

    % Figure 2
    \begin{tikzpicture}
    \begin{axis}[
        scale = .8,
        title = {Mujoco Environments: Actor Loss},
        xlabel = {Episode},
        ylabel = {Actor Loss},
        xmin = 0, xmax = 3200,
        ymin = -200, ymax = 0,
        xtick = {0,800,1600,2400,3200},
        ytick = {0,-25,-50,-75,-100,-125,-150,-175,-200},
        legend pos = north east,
        ymajorgrids = true,
        grid = major,
        mark size = .5pt,
        axis line style = ultra thin,
        legend style = {nodes = {scale = 0.75, transform shape}},
        legend entries = {$Walker Run$, $Cheetah$, $Walker Walk$, $Humanoid Walk$, $Quadruped$,
                        $Walker Stand$, $Cartpole Balance$, $Pendulum Swingup$}
    ]

    \addplot table [x = episode, y = actor_loss, col sep = comma]{experiments/203408-walker-run-train-control.csv};
    \addplot table [x = episode, y = actor_loss, col sep = comma]{others/031508-cheetah-run-train.csv};
    \addplot table [x = episode, y = actor_loss, col sep = comma]{others/154418-walker-walk-train.csv};
    \addplot table [x = episode, y = actor_loss, col sep = comma]{others/175732-humanoid-walk-train.csv};
    \addplot table [x = episode, y = actor_loss, col sep = comma]{others/180549-quadruped-train.csv};
    \addplot table [x = episode, y = actor_loss, col sep = comma]{others/190918-walker-stand-train.csv};
    \addplot table [x = episode, y = actor_loss, col sep = comma]{others/191540-cartpole-balance-train.csv};
    \addplot table [x = episode, y = actor_loss, col sep = comma]{others/201234-pendulum-swingup-train.csv};

    \end{axis}
    \end{tikzpicture}


        This visualization demonstrates that walker run is the best balance between convergence and difficulty.\ Other environments
        such as Cheetah and Pendulum Swingup converge too rapidly to test and tune hyper parameters in a meaningful way.\ As a
        result we used walker run as our control environment and began to test different hyperparameters.\ We tested walker run
        on different learning rates, optimizers, and standard deviation schedulers.\

    \section{Results}\label{sec:results}

    % Figure 3
    \begin{tikzpicture}
    \begin{axis}[
        scale = .8,
        title = {Walker Run Hyperparameter Tuning: Reward},
        xlabel = {Episode},
        ylabel = {Reward},
        xmin = 0, xmax = 3200,
        ymin = 0, ymax = 800,
        xtick = {0,800,1600,2400,3200},
        ytick = {0,200,400,600,800},
        legend pos = south east,
        ymajorgrids = true,
        grid = major,
        mark size = .7pt,
        axis line style = ultra thin,
        legend style = {nodes = {scale = 0.9, transform shape}},
        legend entries = {$Control$, $Step-Linear$, $Lr:1e-2$, $Lr:1e-3$, $Adamw$}
    ]

    \addplot table [x = episode, y = episode_reward, col sep = comma]{experiments/203408-walker-run-eval-control.csv};
    \addplot table [x = episode, y = episode_reward, col sep = comma]{experiments/184403-walker-run-eval-stepliner.csv};
    \addplot table [x = episode, y = episode_reward, col sep = comma]{experiments/215829-walker-run-eval-1e-2.csv};
    \addplot table [x = episode, y = episode_reward, col sep = comma]{experiments/214858-walker-run-eval-1e-3.csv};
    \addplot table [x = episode, y = episode_reward, col sep = comma]{experiments/222844-walker-run-eval-Adamw.csv};

    \end{axis}
    \end{tikzpicture}

    % Figure 4
    \begin{tikzpicture}
    \begin{axis}[
        scale = .75,
        title = {Walker Run Hyperparameter Tuning: Actor Loss},
        xlabel = {Episode},
        ylabel = {Actor Loss},
        xmin = 0, xmax = 3200,
        ymin = -165, ymax = 0,
        xtick = {0,800,1600,2400,3200},
        ytick = {0,-25,-50,-75,-100,-125,-150,-175},
        legend pos = north east,
        ymajorgrids = true,
        grid = major,
        mark size = .5pt,
        axis line style = ultra thin,
        legend style = {nodes = {scale = 0.9, transform shape}},
        legend entries = {$Control$, $StepLinear$, $Lr:1e-2$, $Lr:1e-3$, $Adamw$}
    ]

    \addplot table [x = episode, y = actor_loss, col sep = comma]{experiments/203408-walker-run-train-control.csv};
    \addplot table [x = episode, y = actor_loss, col sep = comma]{experiments/184403-walker-run-train-stepliner.csv};
    \addplot table [x = episode, y = actor_loss, col sep = comma]{experiments/215829-walker-run-train-1e-2.csv};
    \addplot table [x = episode, y = actor_loss, col sep = comma]{experiments/214858-walker-run-train-1e-3.csv};
    \addplot table [x = episode, y = actor_loss, col sep = comma]{experiments/222844-walker-run-train-Adamw.csv};

    \end{axis}
    \end{tikzpicture}

    Figures 3 and 4 demonstrate that this ablation study could not achieve better performance than the initial control model, in
    the long term.\ We find that adjusting the learning rates destroyed the tractability of our model as it was unable to learn
    during its training.\ We tested a total of three learning rates, including a learning rate scheduler which did not run despite
    using the HPC resources.\ We believe this is the result of using learning rates that were too large for the environment space.\
    Lowering these may have provided better results.\

    With respect to the optimizer, AdamW, we found that our model performed slightly worse than the control runs using the Adam
    optimizer.\ We believe this discrepancy is due AdamW being more robust and thus more conservative in its approach to learning.\
    This results in AdamW gaining faster convergence for the first few hundred runs and inevitably causes AdamW to perform
    slightly worse in the long term.\ With enough time however, we would expect to find that AdamW converges to the same rewards
    as the control even if it needs more time.\

    The final hyperparameter we tuned was that of the standard deviation scheduler.\ The theory describes that as learning
    progresses in the policy, the variation should become smaller.\ This is because over time we are more likely to find a more
    optimal policy.\ This allows the reinforcement of these optimal policies by diminishing the variations between training runs.\
    In our experiments we changed the scheduler by decreasing the variation as the policy progressed.\  We find that these changes
    improved convergence time, but did not achieve final accuracies better than that of our control run.\ Despite this, we
    believe this hyper parameter is the least optimized mechanism in the system.\

    We find that some of our experiments were able to converge to a nearly optimal policy quicker than that of the
    control.\ However, as the training continues, the hyper parameters proposed by the DrQ-v2 researchers converge to an optimal
    quicker.\ We believe that with better hardware and updated software on SJSU's HPC we may have been able to see strictly better
    convergence and training times with our proposed hyperparameter adjustments.\ Despite our limited resources, we were able to
    briefly improve convergence times and demonstrate some of the many hyper parameters which can be tuned in DrQ-v2.\


    \section{Conclusion}\label{sec:conclusion}

    This ablation study provided meaningful insight into the mechanisms that make DrQ-v2 a state-of-the-art model.\ The efficiency
    and ingenuity of this model is necessary to provide the significant breakthrough in direct pixel observation we see.\ In all,
    while our experiments to improve the DrQ-v2 failed, we found potential areas of interest, such as the standard deviation
    scheduler, which may offer significant improvements.\ Through a culmination of many optimizations, DrQ-v2 has become a
    reinforcement learning tool capable of democratizing the field.\ Its lightweight architecture and fast training times makes
    this algorithm a gold standard for future reinforcement learning work to build from.\ Future work may find significant success
    in the development of a strong learning rate scheduler and improvements to the standard deviation scheduler.\

    \bibliography{main}
    \bibliographystyle{ieeetr}
\end{document}